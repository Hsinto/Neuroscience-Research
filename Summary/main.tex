\documentclass{article}
\usepackage{graphicx} % Required for inserting images
\usepackage[english]{babel} % English formatting
\usepackage[utf8]{inputenc} % Standard encoding
\usepackage[a4paper,left=4cm,bottom=2cm]{geometry} % Page formatting
\usepackage{indentfirst} % Indents the first paragraph
\usepackage{amsmath} % Maths type package
\usepackage{bm} % Bold font maths
\usepackage{graphicx} % Advanced graphics package
\usepackage[export]{adjustbox} 
\usepackage{fancyhdr} % Fancy headers
\usepackage{hyperref}
\usepackage{amsthm}
\usepackage{cleveref}
\usepackage{bm}
\usepackage{wrapfig} % Text flowing around figures

\newtheorem{thm}{Theorem}[section]
\newtheorem{cor}[thm]{Corollary}
\newtheorem{prop}[thm]{Proposition}
\newtheorem{lem}[thm]{Lemma}
\newtheorem{conj}[thm]{Conjecture}
\newtheorem{quest}[thm]{Question}
\newtheorem{claim}[thm]{Claim}
\newtheorem{ppty}[thm]{Property}

\theoremstyle{definition}
\newtheorem{defn}[thm]{Definition}
\newtheorem{defns}[thm]{Definitions}
\newtheorem{con}[thm]{Construction}
\newtheorem{exmp}[thm]{Example}
\newtheorem{exmps}[thm]{Examples}
\newtheorem{notn}[thm]{Notation}
\newtheorem{notns}[thm]{Notations}
\newtheorem{addm}[thm]{Addendum}
\newtheorem{exer}[thm]{Exercise}
\newtheorem{limit}[thm]{Limitation}


\theoremstyle{remark}
\newtheorem{rem}[thm]{Remark}
\newtheorem{rems}[thm]{Remarks}
\newtheorem{warn}[thm]{Warning}
\newtheorem{sch}[thm]{Scholium}

\pagestyle{fancy}
\fancyhf{}
\renewcommand{\footrulewidth}{0.4pt}

\title{Random CTLNs}
\author{Eric Han, Caitlin Lienkaemper}
\date{July-August 2024}

\begin{document}

\maketitle

\section{Project Summary}
The goal of this project was to observe the effects of symmetry in random combinatorial linear-threshold networks (CTLNs). The simulation model was built in Python. A major step taken this summer was the completion of a small but complete program to generate averaged heatmaps of how network dynamics behaved across various probability parameters. Additionally, at the tail-end of the project, a more mathematically focused thrust to the problem was conceptualized and a more concrete vision of tackling the symmetry problem was realized.

\section{Model}
The CTLN model is a simplified mathematical model of neural networks that focuses on the fine connectivity between neurons as the key factor affecting the resulting dynamics of the network. CTLNs exhibit strongly nonlinear dynamics, showcasing behavior such as multistability, chaotic attractors, and limit cycles. Physically, these models are similar to other classical attractor neural networks that model associative memory. For instance, the manifestation of limit cycles can indicate central pattern generators controlling periodic behaviors in animals. The main method of analysis for the CTLN model is through a graph theoretical lens. One can look directly at the structure and specific connectivity of the graph to determine the underlying dynamics of the network.



\section{Code}
\subsection{Algorithm}
The goal of this model is to generate random graphs with set parameters for symmetry and edge connection probability. This is done through the generation of a random square adjancency matrix of size $n$. Only the top diagonal of the matrix is generated with values of $\{-2,-1,1,2\}$. These values denote no edge, a unidirectional edge from $j$ to $i$, a unidirectional edge from $i$ to $j$, and a bidirectional edge between $j$ and $i$, respectively. The respective weight table for the values is $\{1-p, q(\frac{1-p}{2} ), q(\frac{1-p}{2} ), pq)\}$, where $p$ denotes edge connection probability and $q$ is the symmetry parameter. As $q$ increases, general symmetry of the graph increases linearly.
\begin{defn}
Let $i,j$ be any two vertices with an edge between them in a directed graph. $i$ and $j$ are considered \textit{symmetric} if there exists a bidirectional edge between them. A graph is considered fully symmetric if every edge in the graph is a bidirectional edge; or if there exist no edges in the graph.
\end{defn}
A consequence of this generation method is that the generation of any bidirectional edge is controlled exclusively by the value $\{2\}$, that is, a bidirectional edge cannot be generated by the creation of 2 unidirectional edges in opposing direction between vertices $i$ and $j$.

\end{document}
